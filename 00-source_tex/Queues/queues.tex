\documentclass{article}
\usepackage[utf8]{inputenc}

\usepackage{geometry}
\geometry{a5paper, portrait, margin=0.5in}

\usepackage{hyperref}
\hypersetup{colorlinks=true, urlcolor=blue}

\pagenumbering{gobble}
\begin{document}

Before attempting the problem, you are \textbf{required} to thoroughly read the reference material - \url{https://www.hackerearth.com/practice/data-structures/queues/basics-of-queues/tutorial/}.

\section*{Statement}

Implement and operate a queue according to input. There are two operations to be implemented.

\begin{enumerate}
    \item Enqueue
    \item Dequeue
\end{enumerate}

\section*{Input}

The first line contains an integer $N$, the number of operations that follow. The operations are formatted as given below

\begin{description}
    \item[Enqueue] $E\ x$ - enqueue $x$.
    \item[Dequeue] $D$ - dequeue an element.
\end{description}

\section*{Output}

For each operation, the output should be as given below.

\begin{description}
    \item[Enqueue] Print the new size of the queue.
    \item[Dequeue] Print two space-separated integers, the deleted element (1 if queue is empty) and the new size of the queue.
\end{description}

\section*{Constraints}

\begin{itemize}
    \item $1 \le N \le 100$
    \item $1 \le x \le 100$
\end{itemize}

\section*{Sample}

\begin{tabular}{l|l}
    \hline
    \hline
    Sample Input & Sample Output \\
    \hline
    \verb+5+ & \verb+1+ \\
    \verb+E 2+ & \verb+2 0+ \\
    \verb+D+ & \verb+1 0+ \\
    \verb+D+ & \verb+1+ \\
    \verb+E 3+ & \verb+3 0+ \\
    \verb+D+ & \verb++ \\
    \hline
\end{tabular}

\end{document}
