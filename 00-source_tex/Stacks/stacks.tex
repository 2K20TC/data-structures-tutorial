\documentclass{article}
\usepackage[utf8]{inputenc}

\usepackage{geometry}
\geometry{a5paper, portrait, margin=0.5in}

\usepackage{hyperref}
\hypersetup{colorlinks=true, urlcolor=blue}

\pagenumbering{gobble}
\begin{document}

Before attempting the problem, you are \textbf{required} to thoroughly read the reference material - \url{https://www.hackerearth.com/practice/data-structures/stacks/basics-of-stacks/tutorial/}.

\section*{Statement}

You are a manager at a restaurant that serves food packages. Each package has a associated cost. The packages are piled up next to the manager. The manager has to handle two types of requests.

\begin{description}
    \item[Customer] When a customer demands a package, the package on the top of the pile is given out and he/she is charged the cost of the package. This reduces the height of the pile by $1$. In case the pile is empty, the customer goes away empty-handed.
    
    \item[Chef] A chef prepares a food package and adds it to the top of the pile, and reports the cost to the manager.
\end{description}

\section*{Input}

The first line contains an integer $R$, the number of requests that follow. A customer request is indicated by an integer $1$ in the line. A chef request is indicated by two space-separated integers $2$ and $C$, the cost of the package.

\section*{Output}

For each customer query, output the price the customer has to pay (the cost of the package) on a new line. If the pile is empty, output "No food" (without the quotes).

\section*{Constraints}

\begin{itemize}
    \item $1 \le Q \le 10^{5}$
    \item $1 \le C \le 10^{7}$
\end{itemize}

\section*{Sample}

\begin{tabular}{l|l}
    \hline
    \hline
    Sample Input & Sample Output \\
    \hline
    \verb+6+ & \verb+No food+ \\
    \verb+1+ & \verb+9+ \\
    \verb+2 5+ & \verb+7+ \\
    \verb+2 7+ & \verb++ \\
    \verb+2 9+ & \verb++ \\
    \verb+1+ & \verb++ \\
    \verb+1+ & \verb++ \\
    \hline
\end{tabular}

\end{document}
