\documentclass{article}
\usepackage[utf8]{inputenc}

\usepackage{geometry}
\geometry{a5paper, portrait, margin=0.5in}

\usepackage{hyperref}
\hypersetup{colorlinks=true, urlcolor=blue}

\pagenumbering{gobble}
\begin{document}

Before attempting the problem, you are \textbf{required} to thoroughly read the reference material - \url{https://www.hackerearth.com/practice/data-structures/arrays/multi-dimensional/tutorial/}.

\section*{Statement}

Given a 2D array A, convert all rows to columns and columns to rows.

\section*{Input}

First line of input contains two space separated integers, $N$ - total rows, $M$ - total columns. The following $N$ lines each contain $M$ space-separated integers, the $i,j^{th}$ element of the array ($A[i][j]$).

\section*{Output}

Print $M$ lines, each containing $N$ space-separated integers, such that the input is transposed.

\section*{Constraints}

\begin{itemize}
    \item $1 \le N \le 10$
    \item $1 \le M \le 10$
    \item $1 \le A[i][j] \le 100$, where $0 \le i < N$ and $0 \le j < M$
\end{itemize}

\section*{Sample}

\begin{tabular}{l|l}
    \hline
    \hline
    Sample Input & Sample Output \\
    \hline
    \verb+3 5+ & \verb+13 9 5+ \\
    \verb+13 4 8 14 1+ & \verb+4 6 12+ \\
    \verb+9 6 3 7 21+ & \verb+8 3 17+ \\
    \verb+5 12 17 9 3+ & \verb+14 7 9+ \\
    \ & \verb+1 21 3+ \\
    \hline
\end{tabular}

\end{document}
