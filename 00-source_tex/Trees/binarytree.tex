\documentclass{article}
\usepackage[utf8]{inputenc}

\usepackage{geometry}
\geometry{a5paper, portrait, margin=0.5in}

\usepackage{hyperref}
\hypersetup{colorlinks=true, urlcolor=blue}

\pagenumbering{gobble}
\begin{document}

Before attempting the problem, you are \textbf{required} to thoroughly read the reference material - \url{https://www.hackerearth.com/practice/data-structures/trees/binary-and-nary-trees/tutorial/}.

\section*{Statement}

Given a binary tree which has T nodes, you need to find the diameter of that binary tree. The diameter of a tree is the number of nodes on the longest path between two leaves in the tree.

\section*{Input}

First line contains two integers, $T$ and $X$, the number of nodes in the tree and value of the root node. The following $2 \times (T - 1)$ lines contain details of the nodes.

Each node is described by two lines. The first line contains a string and the second contains an integer, which denote the path to the node from root and the value of the node respectively.

The string consists of the characters $L$ and $R$ only. $L$ denotes left child and $R$ denotes right child.

\section*{Output}

Print the diameter of the binary tree.

\section*{Constraints}

\begin{itemize}
    \item $1 \le T \le 20$
    \item $1 \le valueofnodes \le 20$
\end{itemize}

\section*{Sample}

\begin{tabular}{l|l}
    \hline
    \hline
    Sample Input & Sample Output \\
    \hline
    \verb+5 1+ & \verb+4+ \\
    \verb+L+ & \verb++ \\
    \verb+2+ & \verb++ \\
    \verb+R+ & \verb++ \\
    \verb+3+ & \verb++ \\
    \verb+LL+ & \verb++ \\
    \verb+4+ & \verb++ \\
    \verb+LR+ & \verb++ \\
    \verb+5+ & \verb++ \\
    \hline
\end{tabular}

\end{document}
