\documentclass{article}
\usepackage[utf8]{inputenc}

\usepackage{geometry}
\geometry{a5paper, portrait, margin=0.5in}

\usepackage{hyperref}
\hypersetup{colorlinks=true, urlcolor=blue}

\pagenumbering{gobble}
\begin{document}

Before attempting the problem, you are \textbf{required} to thoroughly read the reference material - \url{https://www.hackerearth.com/practice/data-structures/trees/binary-search-tree/tutorial/}.

\section*{Statement}

Create a binary search tree from list $A$ containing $N$ elements, inserting elements in the same order as the input. Print the pre-order traversal of the sub-tree with root node data equal to $Q$ (inclusive of $Q$), separating each element by a space.

\section*{Input}

The first line contains a single integer $N$, the number of elements. The second line contains $N$ space-separated integers. The third line contains a single integer $Q$, the element whose sub-tree is to be printed in pre-order form. 

\section*{Output}

Print $K$ integers (each on a new line), where $K$ is the number of elements in the subtree of $Q$ ($Q$ inclusive).

\section*{Constraints}

\begin{itemize}
    \item $1 \le N \le 10^{3}$
    \item $-10^{9} \le valueofnodes \le 10^{9}$
\end{itemize}

\section*{Sample}

\begin{tabular}{l|l}
    \hline
    \hline
    Sample Input & Sample Output \\
    \hline
    \verb+4+ & \verb+3+ \\
    \verb+2 1 3 4+ & \verb+4+ \\
    \verb+3+ & \verb++ \\
    \hline
\end{tabular}

\end{document}
