\documentclass{article}
\usepackage[utf8]{inputenc}

\usepackage{geometry}
\geometry{a5paper, portrait, margin=0.5in}

\usepackage{hyperref}
\hypersetup{colorlinks=true, urlcolor=blue}

\pagenumbering{gobble}
\begin{document}

Before attempting the problem, you are \textbf{required} to thoroughly read the reference material - \url{https://www.hackerearth.com/practice/data-structures/trees/heapspriority-queues/tutorial/}.

\section*{Statement}

You have been given a sequence $A$ of $N$ digits. Each digit in this sequence ranges from $1$ to $10^{9}$. You need to perform two types of operations on this list.

\begin{description}
    \item[Addition] $Add(x)$, add an element x to the end of the list.
    \item[Maximum of list] $Max(list)$, find the maximum element of the current sequence.
\end{description}

\section*{Input}

The first line consists of a single integer $N$, the size of the initial sequence. The next line consists of $N$ space separated integers representing the elements of the initial sequence.

The next line contains a single integer $q$, the number of queries. The next $q$ lines contain the operation details. The first integer type indicates the type of query. If $type_{i} = 1$, it is followed by another integer $x$ and the operation to be performed is addition. If $type_{i} = 2$, the other operation is to be performed.

\section*{Output}

For each operation of the second type, print a single integer on a new line. 

\section*{Constraints}

\begin{itemize}
    \item $1 \le N \le 10^{5}$
    \item $1 \le A[i] \le 10^{9}$
    \item $1 \le q \le 10^{4}$
\end{itemize}

\section*{Sample}

\begin{tabular}{l|l}
    \hline
    \hline
    Sample Input & Sample Output \\
    \hline
    \verb+5+ & \verb+5+ \\
    \verb+1 2 3 4 5+ & \verb++ \\
    \verb+4+ & \verb++ \\
    \verb+1 1+ & \verb++ \\
    \verb+1 2+ & \verb++ \\
    \verb+1 3+ & \verb++ \\
    \verb+2+ & \verb++ \\
    \hline
\end{tabular}

\end{document}
